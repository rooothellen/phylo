% Options for packages loaded elsewhere
\PassOptionsToPackage{unicode}{hyperref}
\PassOptionsToPackage{hyphens}{url}
\documentclass[
]{article}
\usepackage{xcolor}
\usepackage[margin=1in]{geometry}
\usepackage{amsmath,amssymb}
\setcounter{secnumdepth}{-\maxdimen} % remove section numbering
\usepackage{iftex}
\ifPDFTeX
  \usepackage[T1]{fontenc}
  \usepackage[utf8]{inputenc}
  \usepackage{textcomp} % provide euro and other symbols
\else % if luatex or xetex
  \usepackage{unicode-math} % this also loads fontspec
  \defaultfontfeatures{Scale=MatchLowercase}
  \defaultfontfeatures[\rmfamily]{Ligatures=TeX,Scale=1}
\fi
\usepackage{lmodern}
\ifPDFTeX\else
  % xetex/luatex font selection
\fi
% Use upquote if available, for straight quotes in verbatim environments
\IfFileExists{upquote.sty}{\usepackage{upquote}}{}
\IfFileExists{microtype.sty}{% use microtype if available
  \usepackage[]{microtype}
  \UseMicrotypeSet[protrusion]{basicmath} % disable protrusion for tt fonts
}{}
\makeatletter
\@ifundefined{KOMAClassName}{% if non-KOMA class
  \IfFileExists{parskip.sty}{%
    \usepackage{parskip}
  }{% else
    \setlength{\parindent}{0pt}
    \setlength{\parskip}{6pt plus 2pt minus 1pt}}
}{% if KOMA class
  \KOMAoptions{parskip=half}}
\makeatother
\usepackage{color}
\usepackage{fancyvrb}
\newcommand{\VerbBar}{|}
\newcommand{\VERB}{\Verb[commandchars=\\\{\}]}
\DefineVerbatimEnvironment{Highlighting}{Verbatim}{commandchars=\\\{\}}
% Add ',fontsize=\small' for more characters per line
\usepackage{framed}
\definecolor{shadecolor}{RGB}{248,248,248}
\newenvironment{Shaded}{\begin{snugshade}}{\end{snugshade}}
\newcommand{\AlertTok}[1]{\textcolor[rgb]{0.94,0.16,0.16}{#1}}
\newcommand{\AnnotationTok}[1]{\textcolor[rgb]{0.56,0.35,0.01}{\textbf{\textit{#1}}}}
\newcommand{\AttributeTok}[1]{\textcolor[rgb]{0.13,0.29,0.53}{#1}}
\newcommand{\BaseNTok}[1]{\textcolor[rgb]{0.00,0.00,0.81}{#1}}
\newcommand{\BuiltInTok}[1]{#1}
\newcommand{\CharTok}[1]{\textcolor[rgb]{0.31,0.60,0.02}{#1}}
\newcommand{\CommentTok}[1]{\textcolor[rgb]{0.56,0.35,0.01}{\textit{#1}}}
\newcommand{\CommentVarTok}[1]{\textcolor[rgb]{0.56,0.35,0.01}{\textbf{\textit{#1}}}}
\newcommand{\ConstantTok}[1]{\textcolor[rgb]{0.56,0.35,0.01}{#1}}
\newcommand{\ControlFlowTok}[1]{\textcolor[rgb]{0.13,0.29,0.53}{\textbf{#1}}}
\newcommand{\DataTypeTok}[1]{\textcolor[rgb]{0.13,0.29,0.53}{#1}}
\newcommand{\DecValTok}[1]{\textcolor[rgb]{0.00,0.00,0.81}{#1}}
\newcommand{\DocumentationTok}[1]{\textcolor[rgb]{0.56,0.35,0.01}{\textbf{\textit{#1}}}}
\newcommand{\ErrorTok}[1]{\textcolor[rgb]{0.64,0.00,0.00}{\textbf{#1}}}
\newcommand{\ExtensionTok}[1]{#1}
\newcommand{\FloatTok}[1]{\textcolor[rgb]{0.00,0.00,0.81}{#1}}
\newcommand{\FunctionTok}[1]{\textcolor[rgb]{0.13,0.29,0.53}{\textbf{#1}}}
\newcommand{\ImportTok}[1]{#1}
\newcommand{\InformationTok}[1]{\textcolor[rgb]{0.56,0.35,0.01}{\textbf{\textit{#1}}}}
\newcommand{\KeywordTok}[1]{\textcolor[rgb]{0.13,0.29,0.53}{\textbf{#1}}}
\newcommand{\NormalTok}[1]{#1}
\newcommand{\OperatorTok}[1]{\textcolor[rgb]{0.81,0.36,0.00}{\textbf{#1}}}
\newcommand{\OtherTok}[1]{\textcolor[rgb]{0.56,0.35,0.01}{#1}}
\newcommand{\PreprocessorTok}[1]{\textcolor[rgb]{0.56,0.35,0.01}{\textit{#1}}}
\newcommand{\RegionMarkerTok}[1]{#1}
\newcommand{\SpecialCharTok}[1]{\textcolor[rgb]{0.81,0.36,0.00}{\textbf{#1}}}
\newcommand{\SpecialStringTok}[1]{\textcolor[rgb]{0.31,0.60,0.02}{#1}}
\newcommand{\StringTok}[1]{\textcolor[rgb]{0.31,0.60,0.02}{#1}}
\newcommand{\VariableTok}[1]{\textcolor[rgb]{0.00,0.00,0.00}{#1}}
\newcommand{\VerbatimStringTok}[1]{\textcolor[rgb]{0.31,0.60,0.02}{#1}}
\newcommand{\WarningTok}[1]{\textcolor[rgb]{0.56,0.35,0.01}{\textbf{\textit{#1}}}}
\usepackage{graphicx}
\makeatletter
\newsavebox\pandoc@box
\newcommand*\pandocbounded[1]{% scales image to fit in text height/width
  \sbox\pandoc@box{#1}%
  \Gscale@div\@tempa{\textheight}{\dimexpr\ht\pandoc@box+\dp\pandoc@box\relax}%
  \Gscale@div\@tempb{\linewidth}{\wd\pandoc@box}%
  \ifdim\@tempb\p@<\@tempa\p@\let\@tempa\@tempb\fi% select the smaller of both
  \ifdim\@tempa\p@<\p@\scalebox{\@tempa}{\usebox\pandoc@box}%
  \else\usebox{\pandoc@box}%
  \fi%
}
% Set default figure placement to htbp
\def\fps@figure{htbp}
\makeatother
\setlength{\emergencystretch}{3em} % prevent overfull lines
\providecommand{\tightlist}{%
  \setlength{\itemsep}{0pt}\setlength{\parskip}{0pt}}
\usepackage{bookmark}
\IfFileExists{xurl.sty}{\usepackage{xurl}}{} % add URL line breaks if available
\urlstyle{same}
\hypersetup{
  pdftitle={parsimony},
  hidelinks,
  pdfcreator={LaTeX via pandoc}}

\title{parsimony}
\author{}
\date{\vspace{-2.5em}2026-02-03}

\begin{document}
\maketitle

\begin{Shaded}
\begin{Highlighting}[]
\FunctionTok{library}\NormalTok{(phangorn)}
\end{Highlighting}
\end{Shaded}

\begin{verbatim}
## Loading required package: ape
\end{verbatim}

\begin{Shaded}
\begin{Highlighting}[]
\NormalTok{primate }\OtherTok{\textless{}{-}} \FunctionTok{read.delim}\NormalTok{(}\StringTok{"exampleData.txt"}\NormalTok{)}
\end{Highlighting}
\end{Shaded}

\begin{Shaded}
\begin{Highlighting}[]
\NormalTok{primate}
\end{Highlighting}
\end{Shaded}

\begin{verbatim}
##           species char1 char2 char3 char4 char5 char6 char7 char8 char9 char10
## 1      chimpanzee     1     1     1     2     2     2     2     3     1      1
## 2   howler_monkey     0     1     0     3     1     3     2     2     2      2
## 3   rhesus_monkey     1     1     1     2     2     2     2     1     1      1
## 4         gorilla     1     1     1     2     3     3     3     3     1      1
## 5 ring_tail_lemur     0     1     0     2     1     1     1     1     2      2
## 6          gibbon     0     1     1     1     2     1     1     2     1      1
## 7           human     0     0     0     3     3     2     3     3     1      1
##   char11 char12 char13 char14 char15
## 1      1      1      2      2      2
## 2      1      1      1      2      2
## 3      1      1      2      3      3
## 4      1      1      3      1      3
## 5      0      1      1      1      1
## 6      0      0      2      3      1
## 7      0      0      3      2      2
\end{verbatim}

\begin{Shaded}
\begin{Highlighting}[]
\NormalTok{primate\_df }\OtherTok{\textless{}{-}} \FunctionTok{data.frame}\NormalTok{(primate[,}\DecValTok{2}\SpecialCharTok{:}\DecValTok{11}\NormalTok{], }\AttributeTok{row.names =}\NormalTok{ primate[,}\DecValTok{1}\NormalTok{])}
\NormalTok{primate.mat }\OtherTok{\textless{}{-}} \FunctionTok{as.matrix}\NormalTok{(primate\_df)}
\NormalTok{levels }\OtherTok{\textless{}{-}} \FunctionTok{c}\NormalTok{(}\StringTok{"0"}\NormalTok{, }\StringTok{"1"}\NormalTok{, }\StringTok{"2"}\NormalTok{, }\StringTok{"3"}\NormalTok{)}
\NormalTok{primate.phy }\OtherTok{\textless{}{-}} \FunctionTok{phyDat}\NormalTok{(primate.mat, }\AttributeTok{type =} \StringTok{"USER"}\NormalTok{, }\AttributeTok{levels =}\NormalTok{ levels)}
\NormalTok{primate.phy}
\end{Highlighting}
\end{Shaded}

\begin{verbatim}
## 7 sequences with 10 character and 9 different site patterns.
## The states are 0 1 2 3
\end{verbatim}

\begin{Shaded}
\begin{Highlighting}[]
\NormalTok{dm }\OtherTok{\textless{}{-}} \FunctionTok{dist.hamming}\NormalTok{(primate.phy)}
\NormalTok{treeUPGMA }\OtherTok{\textless{}{-}} \FunctionTok{upgma}\NormalTok{(dm)}
\end{Highlighting}
\end{Shaded}

\begin{Shaded}
\begin{Highlighting}[]
\NormalTok{tree\_parsimony }\OtherTok{\textless{}{-}} \FunctionTok{optim.parsimony}\NormalTok{(treeUPGMA, primate.phy)}
\end{Highlighting}
\end{Shaded}

\begin{verbatim}
## Final p-score 21 after  1 nni operations
\end{verbatim}

\textbf{Question 1}: It runs very quickly (\textless{} 1 second).
Returns a p-score of 21 after 1 nni operation.

\begin{Shaded}
\begin{Highlighting}[]
\FunctionTok{plot.phylo}\NormalTok{(tree\_parsimony)}
\end{Highlighting}
\end{Shaded}

\pandocbounded{\includegraphics[keepaspectratio]{3_parsimony_files/figure-latex/unnamed-chunk-7-1.pdf}}

\textbf{Question 3}: The root taxon has three branches (it is not
properly bifurcated).

\begin{Shaded}
\begin{Highlighting}[]
\NormalTok{tree\_parsimony }\OtherTok{\textless{}{-}} \FunctionTok{root}\NormalTok{(tree\_parsimony, }
                       \AttributeTok{outgroup =} \StringTok{"ring\_tail\_lemur"}\NormalTok{, }
                       \AttributeTok{resolve.root =} \ConstantTok{TRUE}\NormalTok{)}

\FunctionTok{plot.phylo}\NormalTok{(tree\_parsimony)}
\end{Highlighting}
\end{Shaded}

\pandocbounded{\includegraphics[keepaspectratio]{3_parsimony_files/figure-latex/unnamed-chunk-8-1.pdf}}

\textbf{Question 5}: Rhesus monkey + chimpanzee + gorilla form a
monophletic clade. All the organisms in this group have:

\begin{itemize}
\tightlist
\item
  a protruding jaw, enlongated canines, a prominent orbital ridge/crest,
  a bony step-off at the back of the skull
\item
  a similar number of cranial vein holes in their skull
\item
  orbital diameter
\item
  number of teeth and pointy molars
\end{itemize}

Among these traits, the ones that are not shared by other organisms
(suggesting a synaptomorphy) include the protruding jaw.

\end{document}
